% Здесь НЕ НУЖНО делать begin document, включать какие-то пакеты..
% Все уже подрубается в головном файле
% Хедер обыкновенный хсе-теха, все его команды будут здесь работать
% Пожалуйста, проверяйте корректность теха перед пушем

% Здесь формулировка билета
\subsection*{38. Сформулируйте и докажите теорему о мере декартова произведения жордановых множеств.}
    \begin{theorem}[О мере декартова произведения жордановых множеств]
        $E \subset \mathbb{R}^k, \, F \subset \mathbb{R}^l$ --- простые множества 
        ($E = \bigsqcup\limits_{i=1}^n Q_i, \, F = \bigsqcup\limits_{j=1}^m P_j$).
        
        Рассмотрим декартово произведение $E \times F = \bigsqcup\limits_{i, j} \left( Q_i \times P_j \right)$ .
        
        Для меры декартова произведения множеств верно:
        \[ \mu(Q_i \times P_j) = \mu(Q_i) \cdot \mu(P_j) \implies \mu(E \times F) = \mu(E) \cdot \mu(F) \]
    \end{theorem}
    \begin{proof}
        Из определения меры: 
        $\mu(E) = \sum\limits_{i=1}^n \mu(Q_i), \; \mu(F) = \sum\limits_{j=1}^m \mu(P_j)$.
        
        $E \times F = \bigsqcup\limits_{i, j} \left( Q_i \times P_j \right) \implies \mu(E \times F) 
        = \sum\limits_{i, j} \mu\left( Q_i \times P_j \right) =\\
        = \sum\limits_{i, j} \mu(Q_i) \cdot \mu(P_j) = \sum\limits_{i} \mu(Q_i) \cdot \sum\limits_{j} \mu(P_j) 
        = \mu(E) \cdot \mu(F)$.
    \end{proof}

